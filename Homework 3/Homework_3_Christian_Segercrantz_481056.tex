\documentclass{article}
\usepackage[margin=3cm]{geometry}
\usepackage[utf8]{inputenc}
\usepackage{amsmath}
\usepackage{amssymb}
\usepackage{float}
\usepackage{enumitem}
\usepackage{graphicx}
\usepackage{caption}
\usepackage{subcaption}

\graphicspath{ {plots/} }

\title{Nonlinear Optimization - Homework 2 }
\author{Christian Segercrantz 481056}


\begin{document}
\maketitle
\pagebreak
\section*{3.1 FJ and KKT Conditions at Optimal Point}
\subsection*{a)}
	\begin{alignat}{2}
		\text{min. } & -x_1 \\
		\text{subject to: } & x_2 \leq (1-x_1)^3 \label{eq:3.1_g1}\\
		&x_1 \geq 0 \\
		& x_2 \geq 0
	\end{alignat}
	\begin{figure}[H]
		\includegraphics[width=0.8\textwidth]{3_1.png}
		\caption{The feasible region of the problem of exercise 3.1. The condition of $x_1$,$x_2\geq 0$ is implemented by the limits of the plot.}
		\label{fig:1a}
	\end{figure}
	Figure \ref{fig:1a} shows the feasible region for the problem above. Since minimizing $-x_1$ is the same as maximizing $x_1$, we can identify the optimal point as $\bar{x} = \begin{bmatrix} 1 \\ 0 \end{bmatrix}$.
\subsection*{b)}
	We will change around Equation \ref{eq:3.1_g1} to be $(1-x_1)^3-x_2 \geq 0$ for it to fit into the FJ conditions. We know that $u_i g_i(\bar{x}) = 0$ for all $i= 1,...,m$. Hence we can calculate $u_i$ for all $i= 1,...,m$ as 
	
	\begin{align}
		u_1 g_1(\bar{x}) &= u_1\cdot (1-1)^3-0 = 0 \implies 0=0 \\
		u_2 g_2(\bar{x}) &= u_2\cdot 1 = 0 \implies u_2 = 0\\
		u_3 g_3(\bar{x}) &= u_3\cdot 0 = 0 \implies 0=0
	\end{align}
	
	\begin{align}
		0 &= u_0 \nabla f(\bar{x}) + \sum_{i=1}^{m} u_i \nabla g_i(\bar{x}) \\
		0 &= u_0 (-1 \begin{bmatrix} 1 \\ 0 \end{bmatrix}) + u_1 \begin{bmatrix}3(1-x_1)^2 \\ -1 \end{bmatrix} + u_2 \begin{bmatrix} 1 \\ 0 \end{bmatrix} + u_3 \begin{bmatrix} 0 \\ 1 \end{bmatrix} \\
		0 &= u_0 ( \begin{bmatrix} -1 \\ 0 \end{bmatrix}) + u_1 \begin{bmatrix} 0 \\ -1 \end{bmatrix} + u_2 \begin{bmatrix} 1 \\ 0 \end{bmatrix} + u_3 \begin{bmatrix} 0 \\ 1 \end{bmatrix} \\
		0 &= \begin{bmatrix}
			-u_0+u_2 \\
			-u_1+u_3
		\end{bmatrix}\\
		\implies & \begin{cases}
			u_0 = u_2  = 0\\
			u_1 = u_3
		\end{cases}		
	\end{align}
	The point $\bar{x}$ is a FJ point, since we can choose $u_1$ and $u_3$ such that FJ conditions are satisfied. U is thus $u= \begin{bmatrix} 0 \\ t \\ 0 \\ t\end{bmatrix},\quad t > 0$.
\subsection*{c)}
	The KKT conditions are
	\begin{alignat}{2}
		&\nabla f(\bar{x}) + \sum_{i=1}^{m} u_i \nabla g_i(\bar{x}) = 0,\\
		& u_i g_i(x) = 0 , & \forall i \\
		& u_i \geq0, & \forall i
	\end{alignat}
	which gives us 
	\begin{align}
		&  \begin{bmatrix} -1 \\ 0 \end{bmatrix} + u_1 \begin{bmatrix} 0 \\ -1 \end{bmatrix} + u_3 \begin{bmatrix} 0 \\ 1 \end{bmatrix} \\
		= & \begin{bmatrix}-1 \\ u_3 - u_1 \end{bmatrix} \neq \begin{bmatrix}0 \\ 0 \end{bmatrix}.
	\end{align}
	Hence, KKT conditons are not satisfied for any $u$.
	
	In order for LIQC to hold, the gradient of all active inequality constraints and all equality constraints needs to be linearly indepdendent. We can clearly see that, at $\bar{x}$, $\nabla g_1(\bar{x}) = \begin{bmatrix} 0 \\ -1 \end{bmatrix}$ and  $\nabla g_3(\bar{x}) = \begin{bmatrix} 0 \\ 1 \end{bmatrix}$ are linearly dependent. 
	
	For Slater's QC to hold, all inequality costraints needs to be convex in the feasible region. Since $g_2$ and $g_3$ are linear functions, we know that they are convex. We will examine the Hessian for $g_1$ to determine it's convexity. 
	\begin{equation}
		H(g_1(x)) = \begin{bmatrix}
			-6(1-x_1) & 0 \\ 0 & 0
		\end{bmatrix}
	\end{equation}
	Since the Hessian for $g_1$ is not positive semi-definite in all of the feasible reagion, e.g. at $\begin{bmatrix} 0\\0	\end{bmatrix}$, Slater's QC are not satisfied.
\section*{3.2  KKT Conditions for a Quadratic Problem}
\subsection*{a)}
	\begin{alignat}{2}
		\text{min. } & (x_1 +\frac{9}{4})^2 + (x_2 - 2)^2 \\
		\text{subject to: } & x_2-x_1^2 \geq 0 \iff x_1^2-x_2 \leq 0\\
		& x_1 + x_2 \leq 6 \iff  x_1 + x_2 - 6 \leq 0\\
		& x_1 \geq 0 \iff -x_1 \leq 0\\
		& x_2 \geq 0 \iff -x_2 \leq 0
	\end{alignat}
	The KKT conditions for the problem is
	\begin{alignat}{2}
		&\nabla f(\bar{x}) + \sum_{i=1}^{m} u_i \nabla g_i(\bar{x}) = 0 \label{eq:3.2a KKT cond 1}\\
		& u_i g_i(x) = 0 , & \forall i \label{eq:3.2a KKT cond 2}\\
		& u_i \geq0, & \forall i
	\end{alignat}
	From \ref{eq:3.2a KKT cond 2} we get the following at $\bar{x}$:
	\begin{align}
		u_1\left(\left(\frac{3}{2}\right)^2 - 9/4\right) &= u_1 \cdot 0 = 0 \\
		u_2\left(\frac{3}{2} + 9/4 - 6 \right) &= -\frac{9}{4}u_2 \implies u_2 = 0
	\end{align}
	\begin{align}
		0 = & \begin{bmatrix} 2(\bar{x}_1 +\frac{9}{4}) \\ 2(\bar{x}_2 -2)\end{bmatrix} + u_1 \begin{bmatrix} -2\bar{x}_1 \\ 1 \end{bmatrix} + u_2 \begin{bmatrix} -1 \\ -1 \end{bmatrix} + u_3 \begin{bmatrix} 1 \\ 0 \end{bmatrix} + u_4 \begin{bmatrix} 0 \\ 1 \end{bmatrix}\\
		0 = & \begin{bmatrix} 2(\frac{3}{2} +\frac{9}{4}) \\ 2(\frac{9}{4} -2)\end{bmatrix} + u_1 \begin{bmatrix} -2\frac{3}{2} \\ 1 \end{bmatrix} + u_2 \begin{bmatrix} -1 \\ -1 \end{bmatrix} + u_3 \begin{bmatrix} 1 \\ 0 \end{bmatrix} + u_4 \begin{bmatrix} 0 \\ 1 \end{bmatrix} \\
		0 = & \begin{bmatrix}2(\frac{3}{2} +\frac{9}{4}) -\frac{7}{2}u_1 - u_2 +u_3 \\ 2(\frac{9}{4} -2) +u_1 -u_2 +u_4
		\end{bmatrix}
	\end{align}
\subsection*{b)}
	\begin{figure}[H]
		\includegraphics[width=0.8\textwidth]{3_2.png}
		\caption{}
		\label{fig:2a}
	\end{figure}
\section*{3.3 Lagrangian Dual of a Least-Squares Problem}
	\begin{alignat}{2}
		\text{min. } & x^\top x \\
		\text{subject to: } & Ax=b	
	\end{alignat}
\end{document}