\documentclass{article}
\usepackage[margin=3cm]{geometry}
\usepackage[utf8]{inputenc}
\usepackage{amsmath}
\usepackage{amssymb}
\usepackage{float}
\usepackage{enumitem}
\usepackage{graphicx}
\usepackage{caption}
\usepackage{subcaption}

\graphicspath{ {plots/} }

\title{Nonlinear Optimization - Homework 4}
\author{Christian Segercrantz 481056}


\begin{document}
	\maketitle
	\pagebreak
\section*{4.1 Frank-Wolfe Method}
\section*{4.2 Interior-Point Method for Quadratic Problems}
	The problem
	\begin{alignat}{2}
		\text{min. } & c^\top x + \frac{1}{2}x^\top Q x \\
		\text{subject to: } & Ax = b \label{eq:4.1_g1_primal}\\
		&x \geq 0
	\end{alignat}
	and it's dual
	\begin{alignat}{2}
		\text{min. } & b^\top v \frac{1}{2}x^\top Q x \\
		\text{subject to: } & A^\top v + u - Qx = c \label{eq:4.1_g1_dual}\\
		&x \geq 0.
	\end{alignat}
\subsection*{a)}
	The KKT conditions for the problem are
	\begin{align}
		0=& \nabla f(x) + \sum_j u_j \nabla g(x) + \sum_i v_i\nabla h(x) \\
		u_jg(x) =& 0 \quad j=1...m \\
		v_ih(x) =& 0 \quad i=1...l \\
		x\in X, g_j(x)\leq& 0 \quad j=1...m \\
		u_j\geq& 0 \quad j=1...m \\
	\end{align}
	which for our problem becomes
	\begin{align}
		\nabla (c^\top x + \frac{1}{2}x^\top Q x) - u_1 + v_1 \nabla(Ax-b) =& 0\\
		c^\top+Qx - u_1 + v_1A =& 0 \\
		u_1^\top x =& 0 \\
		v_1(Ax-b) =& 0  \\
		x \geq& 0 \\
		u_i\geq& 0  \\
	\end{align}
\subsection*{b)}
	The general form of the Barrier problem looks as
	\begin{alignat}{2}
		\text{(BP): } & \inf_\mu \theta(\mu) \\
		\text{subject to: } & \mu > 0\\
	\end{alignat}
	where $\theta(\mu)$ is, in our case,
	\begin{align}
		\theta(\mu)=& \inf_x \{ f(x) + \mu B(x): -x \leq 0\}\\
		=& \inf_x \{ f(x) + \mu (-\sum_{j=1}^{m}\ln(-g_j(x))): -x \leq 0\} \\
		=& \inf_x \{ f(x) + \mu (-\ln(x)): -x \leq 0\} \\
		=& \inf_x \{ f(x) - \mu \ln(x): -x \leq 0\}.
	\end{align} 
	The problem can thus be formulated as
	\begin{alignat}{2}
		\text{(BP): } & \inf_\mu \{\inf_x \{ c^\top x + \frac{1}{2}x^\top Q x - \mu \ln(x): -x \leq 0\}\} \\
		\text{subject to: } & \mu > 0.
	\end{alignat}
\subsection*{c)}
	The first part of the KKT conditions of part b) is
	\begin{align}
		c^\top+Qx - u_1 + v_1A -\frac{\mu}{x} &= 0
	\end{align}
	and the rest follows as from the a) part.
	
	From the lecture slides and using our problem we get the following Newton system
	\begin{equation}
		\begin{bmatrix}
			A & 0 & 0\\
			-Q & A^\top & I\\
			\bar{U}^k & 0 & \bar{X}^k
		\end{bmatrix}
		\begin{bmatrix}
			d_x^{k+1}\\
			d_v^{k+1}\\
			d_u^{k+1}
		\end{bmatrix} =
		\begin{bmatrix}
			Ax^{k}-b \\
			A^\top x^k+ u^k -Qx^k-c\\
			X^k U^k e - \mu^{k+1}e
		\end{bmatrix} = -
		\begin{bmatrix}
			r_p\\
			r_d\\
			r_c
		\end{bmatrix},
	\end{equation}
	where the first row of the matrix after the first equality sign is the condition of the primal problem and the second is the condition of the dual problem. From this we get the following system of equations
	\begin{equation}
		\begin{cases}
			Ad_x^{k+1} = Ax^{k}-b = -r_p \\
			-Qd_x^{k+1}+A^\top d_v^{k+1} + Id_u = A^\top x^k+ u^k -Qx^k-c = -r_d\\
			\bar{U}^kd_x + \bar{X}^kd_u= X^k U^k e - \mu^{k+1}e = -r_c
		\end{cases}.
	\end{equation}
	From the second expression we can solve $d_u$ as 
	\begin{align}
		A^\top d_v^{k+1} + Id_u &= -r_d\\
		d_u &= -r_d +Qd_x^{k+1}- A^\top d_v^{k+1}.
	\end{align}
	By substituting this into third expression we get
	\begin{align}
		\bar{U}^kd_x + \bar{X}^kd_u &= -r_c\\
		\bar{U}^kd_x - \bar{X}^k(r_d +Qd_x^{k+1}- A^\top d_v^{k+1}) &= -r_c\\
		\bar{U}^kd_x - \bar{X}^kQd_x^{k+1} - \bar{X}^kr_d +\bar{X}^kA^\top d_v^{k+1} &= -r_c\\
		(\bar{U}^k - \bar{X}^kQ)d_x^{k+1} &= -r_c + \bar{X}^kr_d -\bar{X}^kA^\top d_v^{k+1}\\
		d_x^{k+1} &= (\bar{U}^k - \bar{X}^kQ) ^{-1}(-r_c + \bar{X}^kr_d -\bar{X}^kA^\top d_v^{k+1})
	\end{align}
	From the first expression we know that	$Ad_x^{k+1}  = -r_p$
\end{document}